\chapter*{Final Insights}
\addcontentsline{toc}{chapter}{\textbf{Final Insights}}
\section{Conclusion}
The development of real-time object detection systems, particularly when dealing with traffic sign detection, is still an open-research field, and poses a multitude of challenges. While it is true that with the advent of new deep learning techniques, great strides have been made in terms of runtime and precision alike, they still remain inaccessible to most developers due to the data and hardware requirements.\\
In this sense, the method that has been developed in this thesis, which is inspired from the Region-Based Convolutional Neural Network model \cite{rcnn} serves as a good first step towards accessible and accurate object detection systems. It can be applied to any type of object detection task given a labeled training set (no bounding-box requirements), a background dataset that can be generated without massive costs and using relatively modest computing power (FloydHub's CPUs clusters were sufficient).\\
That being said, there are still a number of improvements that can be added to our existing model. In the next section, some of these improvements are briefly detailed.
\section{Future Considerations}
\textbf{Residual modules}, as defined by Xiangyu Zhang et al.\cite{resNet}, introduce skip or shortcut connections which allow the network to use multiple scales making it easier for the layers to learn the identity mappings. It has been demonstrated empirically to benefit the accuracy of the networks and smooth the training process.\\
\textbf{The OverFeat Method} described by Pierre Sermanet, Yann LeCun et al. \cite{overfeat} is a way of increasing performance speed. It works by stripping the final fully connected layers from the network and passing the entire picture as an input (as seen from a car's dashboard camera, in our case) and outputting a feature map for the entire image. This is mathematically equivalent to running a sliding window with strides equal to the sum of the strides in the network's layers and passing them one by one as inputs. This is useful when wanting to use the network as a feature extractor, like in our case.\\
\textbf{Multithreading} implementations take advantage of the modern CPU multi-core architectures by running multiple tasks in parallel. In our case, we could adapt our scripts to use multiple threads for different parts of the image independently. Additionally, we can reduce the workload by ignoring the bottom, top and centermost part of the image since signs are highly likely to appear on either the left or the right side (with a few exceptions).